\documentclass[11pt,letterpaper]{article}

\usepackage[margin=1in]{geometry}
\usepackage{termcal}
\usepackage{enumitem}
\usepackage{hyperref}
\usepackage{color}

\newcommand{\todo}[1]{\textcolor{red}{TODO: #1}}

\title{AA236B: Spacecraft Design Lab}
\author{Zac Manchester}
\date{Winter 2019}

\begin{document}

\maketitle

\section*{Course Description}

Spacecraft design is a truly interdisciplinary subject that draws from every branch of engineering. This course is the laboratory component of AA236 in which students engage in focused hands-on design, fabrication, and testing of small satellite subsystems.

\section*{Instructors}

\begin{center}
\begin{tabular}{l l r}
	Prof. Zac Manchester & \textbf{Email:} \href{mailto:zacmanchester@stanford.edu}{zacmanchester@stanford.edu} & \textbf{Office:} Durand 267 \\
	CA: Walter Maier & \textbf{Email:} wmaier@stanford.edu
\end{tabular}
\end{center}

\section*{Course Objectives and Learning Outcomes}

The goal of this course is to give students hands-on experience designing and building small spacecraft subsystems and integrating them into a CubeSat. Throughout this course, students will:
\begin{enumerate}
	\item Design subsystems to meet system-level engineering requirements.
	\item Work with a small team to fabricate and test spacecraft subsystems.
	\item Coordinate with other teams to integrate subsystems into a complete spacecraft.
\end{enumerate}


\section*{Logistics}

The course will involve designing and building hardware in small teams. There will not be regular lectures. Class time will be used for weekly team meetings and consulting time for teams to meet with the instructors.

\begin{itemize}
	\item Class will be held from 4:30 to 6:20 on Monday and Wednesday in Skilling.
	\item Attendance of weekly team meetings with the instructors is mandatory.
	\item Slack will be used for coordination between teams and instructors. All students will be added to the "SpacecraftDesignLab" slack channel.
	\item GitHub will be used to manage project files for all teams.
\end{itemize}

\section*{Assignments and Exams}

There will be no exams in this course. Evaluation will be based on participation, contribution to design and fabrication work, and a final report from each team documenting their final design.

\section*{Schedule}

\begin{enumerate}[label=\textbf{Week \arabic*:},leftmargin=3.5\parindent]
	\item Team Formation / Introduction
	\item Subsystem Requirements Definition
	\item Preliminary Design Review
	\item Component Purchases
	\item Subsystem Fabrication
	\item Subsystem Fabrication
	\item Subsystem Fabrication
	\item Subsystem Testing
	\item System Integration
	\item System-Level Testing
\end{enumerate}

\section*{Grading}

Grading will be based on:
\begin{itemize}
	\item 40\% Participation and attendance of team meetings
	\item 60\% Completeness, consistency, and quality of the design project
\end{itemize}
Stanford's grading system is defined by the Faculty Senate as A=Excellent, B=Good, C=Satisfactory, D=Minimal Pass, and NP=Not Passed.

\section*{References}

There is no required text for this class. However, students may wish to refer to: \textit{Spacecraft Mission Engineering: The New SMAD} by Wertz, Everett, and Pushcell.

\section*{Course Policies}

\textbf{Attendance:} This is a team-based course. In order to coordinate work among teams, participation in weekly meetings is required. If you are unable to be present at a meeting, you must notify the instructors and ensure that your teammates are prepared to present your work.

\section*{University Policies}

\textbf{The Honor Code:} It is expected that Stanford's Honor Code will be followed in all matters relating to this course. You are encouraged to meet and exchange ideas with your classmates while studying and working on homework assignments, but you are individually responsible for your own work and for understanding the material. You are not permitted to copy or otherwise reference another student's homework or computer code. If you have any questions regarding this policy, feel free to contact the professor.

Compromising your academic integrity may lead to serious consequences, including (but not limited to) one or more of the following: failure of the assignment, failure of the course, disciplinary probation, suspension from the university, or dismissal from the university.

Students are responsible for understanding the University's Honor Code policy and must make proper use of citations of sources for writing papers, creating, presenting, and performing their work, taking examinations, and doing research.

\medskip
\noindent
\textbf{Accommodations:} Students who may need an academic accommodation based on the impact of a disability must initiate the request with the Office of Accessible Education (OAE). Professional staff will evaluate the request with required documentation, recommend reasonable accommodations, and prepare an Accommodation Letter for faculty dated in the current quarter in which the request is being made. Students should contact the OAE as soon as possible since timely notice is needed to coordinate accommodations.




\end{document}
